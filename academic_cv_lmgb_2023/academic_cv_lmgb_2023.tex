%!TEX TS-program = xelatex
%!TEX encoding = UTF-8 Unicode
% Awesome CV LaTeX Template for CV/Resume
%
% This template has been downloaded from:
% https://github.com/posquit0/Awesome-CV
%
% Author:
% Claud D. Park <posquit0.bj@gmail.com>
% http://www.posquit0.com
%
%
% Adapted to be an Rmarkdown template by Mitchell O'Hara-Wild
% 23 November 2018
%
% Template license:
% CC BY-SA 4.0 (https://creativecommons.org/licenses/by-sa/4.0/)
%
%-------------------------------------------------------------------------------
% CONFIGURATIONS
%-------------------------------------------------------------------------------
% A4 paper size by default, use 'letterpaper' for US letter
\documentclass[11pt,a4paper,]{awesome-cv}

% Configure page margins with geometry
\usepackage{geometry}
\geometry{left=1.4cm, top=.8cm, right=1.4cm, bottom=1.8cm, footskip=.5cm}


% Specify the location of the included fonts
\fontdir[fonts/]

% Color for highlights
% Awesome Colors: awesome-emerald, awesome-skyblue, awesome-red, awesome-pink, awesome-orange
%                 awesome-nephritis, awesome-concrete, awesome-darknight

\colorlet{awesome}{awesome-red}

% Colors for text
% Uncomment if you would like to specify your own color
% \definecolor{darktext}{HTML}{414141}
% \definecolor{text}{HTML}{333333}
% \definecolor{graytext}{HTML}{5D5D5D}
% \definecolor{lighttext}{HTML}{999999}

% Set false if you don't want to highlight section with awesome color
\setbool{acvSectionColorHighlight}{true}

% If you would like to change the social information separator from a pipe (|) to something else
\renewcommand{\acvHeaderSocialSep}{\quad\textbar\quad}

\def\endfirstpage{\newpage}

%-------------------------------------------------------------------------------
%	PERSONAL INFORMATION
%	Comment any of the lines below if they are not required
%-------------------------------------------------------------------------------
% Available options: circle|rectangle,edge/noedge,left/right

\name{Luigui}{Gallardo-Becerra}

\position{Data Scientist \& Bioinformatician}

\mobile{+1 619 602 0725}
\email{\href{mailto:luiguimichelgallardo@gmail.com}{\nolinkurl{luiguimichelgallardo@gmail.com}}}
\homepage{luiguigallardo.github.io/web\_page/}
\github{LuiguiGallardo}
\linkedin{luiguigallardo}

% \gitlab{gitlab-id}
% \stackoverflow{SO-id}{SO-name}
% \skype{skype-id}
% \reddit{reddit-id}

\quote{Data Scientist and Bioinformatician looking for new challenges to
apply and expand my skills. In recent years, I participated in different
projects using modern Genomic technologies (NGS), Data Analysis, and
Data Science, to obtain novel biological information that culminated in
high-impact peer-reviewed scientific publications}

\usepackage{booktabs}

\providecommand{\tightlist}{%
	\setlength{\itemsep}{0pt}\setlength{\parskip}{0pt}}

%------------------------------------------------------------------------------



% Pandoc CSL macros
\newlength{\cslhangindent}
\setlength{\cslhangindent}{1.5em}
\newlength{\csllabelwidth}
\setlength{\csllabelwidth}{2em}
\newenvironment{CSLReferences}[3] % #1 hanging-ident, #2 entry spacing
 {% don't indent paragraphs
  \setlength{\parindent}{0pt}
  % turn on hanging indent if param 1 is 1
  \ifodd #1 \everypar{\setlength{\hangindent}{\cslhangindent}}\ignorespaces\fi
  % set entry spacing
  \ifnum #2 > 0
  \setlength{\parskip}{#2\baselineskip}
  \fi
 }%
 {}
\usepackage{calc}
\newcommand{\CSLBlock}[1]{#1\hfill\break}
\newcommand{\CSLLeftMargin}[1]{\parbox[t]{\csllabelwidth}{\honortitlestyle{#1}}}
\newcommand{\CSLRightInline}[1]{\parbox[t]{\linewidth - \csllabelwidth}{\honordatestyle{#1}}}
\newcommand{\CSLIndent}[1]{\hspace{\cslhangindent}#1}

\begin{document}

% Print the header with above personal informations
% Give optional argument to change alignment(C: center, L: left, R: right)
\makecvheader

% Print the footer with 3 arguments(<left>, <center>, <right>)
% Leave any of these blank if they are not needed
% 2019-02-14 Chris Umphlett - add flexibility to the document name in footer, rather than have it be static Curriculum Vitae


%-------------------------------------------------------------------------------
%	CV/RESUME CONTENT
%	Each section is imported separately, open each file in turn to modify content
%------------------------------------------------------------------------------



\hypertarget{experience}{%
\section{Experience}\label{experience}}

\begin{cventries}
    \cventry{}{Graduate Research Assistant (Data Science \& Bioinformatics)}{Institute of Biotechnology, UNAM}{January 2019 - December 2022}{}\vspace{-4.0mm}
\end{cventries}

I was part of different research projects, from the beginning obtaining
or creating the raw data to the final report or publication. The tasks
that I carried out includes the following:

\begin{itemize}
\tightlist
\item
  Management of research projects: participated in the creation of main
  and secondary objectives and research questions from different
  projects to obtain novel biological information useful for the
  scientific community
\item
  Server management (HPC): managed the Laboratory's server, installed
  Linux operative systems and specific software/packages, avoided server
  slowdowns, and built secure server environments
\item
  Creation of pipelines: created pipelines to avoid repetitive workflows
  and make reproducible analysis using bash, Python, and Snakemake
\item
  Software development: developed several software tools to solve and
  achieve specific goals using bash, Python, Perl, and R
\item
  Creation of plots: using R, Rstudio, Jupyter Notebooks, Excel,
  Graphpad, and Tableau
\item
  Maintenance of version control (CI/CD): creation and maintenance of
  GitHub repositories and their continuous version control
\item
  Creation of reports/publications: using Markdown and Word, I delivered
  weekly and biannual reports to the members of the Laboratory. Also, I
  co-authored peer-reviewed publications that are highly cited in their
  respective areas
\item
  Presentations to the general public and academic events: participated
  in scientific dissemination, specialized seminars, and congresses
\end{itemize}

\begin{cventries}
    \cventry{}{Data Analyst}{Appen}{October 2016 - December 2018}{}\vspace{-4.0mm}
\end{cventries}

I was an independent contractor and participated in several projects
that included Data Sourcing and Annotation for different Appen clients
(browsers, social networks, etc.). The tasks I performed were the
following:

\begin{itemize}
\tightlist
\item
  Data collection and Preprocessing: obtention of raw data and
  preprocessing before passing it to the final client or the Machine
  Learning Department
\item
  Creation of weekly deliverables: for the final client or project
  manager
\item
  Content translation: from Spanish to English or English to Spanish
\end{itemize}

\begin{cventries}
    \cventry{}{Software Engineer (Internship)}{Dept. of Computer Science, CUCEI}{May 2016 - July 2016}{}\vspace{-4.0mm}
\end{cventries}

I participated in the creation of a web application. The tasks that I
carried out included:

\begin{itemize}
\tightlist
\item
  Creation of web app: using ASP.NET Core (C\#) and Angular (JavaScript,
  HTML, and CSS), I created an API to save patients' records and images.
\end{itemize}

\newpage

\begin{cventries}
    \cventry{}{Data Scientist \& Bioinformatician (Internship)}{Institute of Biotechnology, UNAM}{January 2016 - July 2016}{}\vspace{-4.0mm}
\end{cventries}

I participated in the research project ``Microbiome of Pacific whiteleg
shrimp reveals differential bacterial community composition between
Wild, Aquacultured and AHPND/EMS outbreak conditions''. The tasks that I
performed were:

\begin{itemize}
\tightlist
\item
  Big Data Analysis: using HPC, Linux, and bash, I analyzed the 16S rRNA
  information of several samples
\item
  Creation of plots: using R, Rstudio, Excel, and Python
\item
  Software development: I developed several tools for ETL data using
  bash, python, and Perl
\item
  Creation of paper: with the data obtained from this internship, I
  co-authored this paper
\end{itemize}

\begin{cventries}
    \cventry{}{Mathematics Teacher (Part-time)}{Student Coaching}{January 2014 - December 2015}{}\vspace{-4.0mm}
\end{cventries}

As a part-time teacher, I taught middle and high school students math
classes to increase their grades or for college admission tests.

\begin{itemize}
\tightlist
\item
  Prepared group and individualized classes and lessons
\item
  Applied weekly assessments to each student
\end{itemize}

\hypertarget{education}{%
\section{Education}\label{education}}

\begin{cventries}
    \cventry{Ph.D. (Computational Biology)}{National Autonomous University of Mexico}{Mexico City, Mexico}{January 2019 - December 2022}{}\vspace{-4.0mm}
\end{cventries}

\begin{cventries}
    \cventry{Master of Science (Computational Biology)}{National Autonomous University of Mexico}{Mexico City, Mexico}{August 2016 - January 2019}{}\vspace{-4.0mm}
\end{cventries}

\begin{cventries}
    \cventry{Bachelor of Science (Molecular Biology)}{University of Guadalajara}{Guadalajara, Mexico}{August 2012 - January 2016}{}\vspace{-4.0mm}
\end{cventries}

\hypertarget{soft-skills}{%
\section{Soft skills}\label{soft-skills}}

\begin{itemize}
\tightlist
\item
  Team player who can also work independently with excellent
  communication skills
\item
  Fast learner and could quickly adapt to new technologies.
\item
  Excellent problem-solving capabilities.
\item
  Can manage a significant workload or several projects from different
  areas without losing track of them.
\end{itemize}

\hypertarget{programing-languages-tools-bioinformatic-skills}{%
\section{Programing Languages, Tools \& Bioinformatic
skills:}\label{programing-languages-tools-bioinformatic-skills}}

\begin{itemize}
\tightlist
\item
  Programing Languages: Python, C\#, R, SQL, Bash, HTML \& CSS,
  JavaScript, Typescript
\item
  Web frameworks: Django, ASP.NET, Angular
\item
  Other tools: High-Performance Computing (HPC), Linux, Tableau, Docker,
  Snakemake, Jupyter Notebooks, Rstudio, MySQL, PostgreSQL, MongoDB
\item
  Bioinformatic skills: 16S rRNA profiling, Genome/Transcriptome
  assembly, and annotation, Metagenomics, Metatranscriptomics, Viromics
\end{itemize}

\hypertarget{languages}{%
\section{Languages}\label{languages}}

\begin{itemize}
\tightlist
\item
  English -- Full professional proficiency
\item
  Spanish -- Native
\end{itemize}

\hypertarget{publications}{%
\section{Publications}\label{publications}}

\hypertarget{bibliography}{}
\leavevmode\vadjust pre{\hypertarget{ref-jatuyospornWhiteSpotSyndrome2023}{}}%
\CSLLeftMargin{1. }
\CSLRightInline{Jatuyosporn, T., Laohawutthichai, P., Romo, J. P. O.,
Gallardo-Becerra, L., Lopez, F. S., Tassanakajon, A., Ochoa-Leyva, A.,
\& Krusong, K. (2023). White spot syndrome virus impact on the
expression of immune genes and gut microbiome of black tiger shrimp
Penaeus monodon. \emph{Scientific Reports}, \emph{13}(1, 1), 996.
\url{https://doi.org/10.1038/s41598-023-27906-8}}

\leavevmode\vadjust pre{\hypertarget{ref-cervantes-echeverriaTwoFacedRoleCrAssphage2023}{}}%
\CSLLeftMargin{2. }
\CSLRightInline{Cervantes-Echeverría, M., Gallardo-Becerra, L.,
Cornejo-Granados, F., \& Ochoa-Leyva, A. (2023). The Two-Faced Role of
crAssphage Subfamilies in Obesity and Metabolic Syndrome: Between Good
and Evil. \emph{Genes}, \emph{14}(1, 1), 139.
\url{https://doi.org/10.3390/genes14010139}}

\leavevmode\vadjust pre{\hypertarget{ref-chinodelacruzCompleteGenomeSequence2023}{}}%
\CSLLeftMargin{3. }
\CSLRightInline{Chino de la Cruz, C. M., Cornejo-Granados, F.,
Gallardo-Becerra, L., Rodríguez-Alegría, M. E., Ochoa-Leyva, A., \&
López Munguía, A. (2023). Complete genome sequence and characterization
of a novel Enterococcus faecium with probiotic potential isolated from
the gut of Litopenaeus vannamei. \emph{Microbial Genomics}, \emph{9}(3),
000938. \url{https://doi.org/10.1099/mgen.0.000938}}

\leavevmode\vadjust pre{\hypertarget{ref-palomino-hermosilloTranscriptomeAnalysisSoursop2022}{}}%
\CSLLeftMargin{4. }
\CSLRightInline{Palomino-Hermosillo, Y. A., Berumen-Varela, G.,
Ochoa-Jiménez, V. A., Balois-Morales, R., Jiménez-Zurita, J. O.,
Bautista-Rosales, P. U., Martínez-González, M. E., López-Guzmán, G. G.,
Cortés-Cruz, M. A., Guzmán, L. F., Cornejo-Granados, F.,
Gallardo-Becerra, L., Ochoa-Leyva, A., \& Alia-Tejacal, I. (2022).
Transcriptome Analysis of Soursop (Annona muricata L.) Fruit under
Postharvest Storage Identifies Genes Families Involved in Ripening.
\emph{Plants}, \emph{11}(14), 1798.
\url{https://doi.org/10.3390/plants11141798}}

\leavevmode\vadjust pre{\hypertarget{ref-ochoa-romoAgavinInducesBeneficial2022}{}}%
\CSLLeftMargin{5. }
\CSLRightInline{Ochoa-Romo, J. P., Cornejo-Granados, F., Lopez-Zavala,
A. A., Viana, M. T., Sánchez, F., Gallardo-Becerra, L., Luque-Villegas,
M., Valdez-López, Y., Sotelo-Mundo, R. R., Cota-Huízar, A.,
López-Munguia, A., \& Ochoa-Leyva, A. (2022). Agavin induces beneficial
microbes in the shrimp microbiota under farming conditions.
\emph{Scientific Reports}, \emph{12}(1, 1), 6392.
\url{https://doi.org/10.1038/s41598-022-10442-2}}

\leavevmode\vadjust pre{\hypertarget{ref-bikelProtocolIsolationSequencing2022}{}}%
\CSLLeftMargin{6. }
\CSLRightInline{Bikel, S., Gallardo-Becerra, L., Cornejo-Granados, F.,
\& Ochoa-Leyva, A. (2022). Protocol for the isolation, sequencing, and
analysis of the gut phageome from human fecal samples. \emph{STAR
Protocols}, \emph{3}(1), 101170.
\url{https://doi.org/10.1016/j.xpro.2022.101170}}

\leavevmode\vadjust pre{\hypertarget{ref-bikelGutDsDNAVirome2021a}{}}%
\CSLLeftMargin{7. }
\CSLRightInline{Bikel, S., López-Leal, G., Cornejo-Granados, F.,
Gallardo-Becerra, L., García-López, R., Sánchez, F., Equihua-Medina, E.,
Ochoa-Romo, J. P., López-Contreras, B. E., Canizales-Quinteros, S.,
Hernández-Reyna, A., Mendoza-Vargas, A., \& Ochoa-Leyva, A. (2021). Gut
dsDNA virome shows diversity and richness alterations associated with
childhood obesity and metabolic syndrome. \emph{iScience}, \emph{24}(8),
102900. \url{https://doi.org/10.1016/j.isci.2021.102900}}

\leavevmode\vadjust pre{\hypertarget{ref-gallardo-becerraMetatranscriptomicAnalysisDefine2020a}{}}%
\CSLLeftMargin{8. }
\CSLRightInline{Gallardo-Becerra, L., Cornejo-Granados, F.,
García-López, R., Valdez-Lara, A., Bikel, S., Canizales-Quinteros, S.,
López-Contreras, B. E., Mendoza-Vargas, A., Nielsen, H., \& Ochoa-Leyva,
A. (2020). Metatranscriptomic analysis to define the Secrebiome, and 16S
rRNA profiling of the gut microbiome in obesity and metabolic syndrome
of Mexican children. \emph{Microbial Cell Factories}, \emph{19}(1), 61.
\url{https://doi.org/10.1186/s12934-020-01319-y}}

\leavevmode\vadjust pre{\hypertarget{ref-cornejo-granadosMetaanalysisRevealsEnvironmental2018}{}}%
\CSLLeftMargin{9. }
\CSLRightInline{Cornejo-Granados, F., Gallardo-Becerra, L.,
Leonardo-Reza, M., Ochoa-Romo, J. P., \& Ochoa-Leyva, A. (2018). A
meta-analysis reveals the environmental and host factors shaping the
structure and function of the shrimp microbiota. \emph{PeerJ}, \emph{6},
e5382. \url{https://doi.org/10.7717/peerj.5382}}

\leavevmode\vadjust pre{\hypertarget{ref-cornejo-granadosMicrobiomePacificWhiteleg2017}{}}%
\CSLLeftMargin{10. }
\CSLRightInline{Cornejo-Granados, F., Lopez-Zavala, A. A.,
Gallardo-Becerra, L., Mendoza-Vargas, A., Sánchez, F., Vichido, R.,
Brieba, L. G., Viana, M. T., Sotelo-Mundo, R. R., \& Ochoa-Leyva, A.
(2017). Microbiome of Pacific Whiteleg shrimp reveals differential
bacterial community composition between Wild, Aquacultured and AHPND/EMS
outbreak conditions. \emph{Scientific Reports}, \emph{7}(1), 11783.
\url{https://doi.org/10.1038/s41598-017-11805-w}}



\end{document}
